%%%%%%%%%%%%%%%%%%%%%%%%%%%%%%%%%%%%%%%%%
\documentclass[11pt,a4paper,roman]{moderncv} % Font sizes: 10, 11, or 12; paper sizes: a4paper, letterpaper, a5paper, legalpaper, executivepaper or landscape; font families: sans or roman

\moderncvstyle{classic} % CV theme - options include: 'casual' (default), 'classic', 'oldstyle' and 'banking'
\moderncvcolor{black} % CV color - options include: 'blue' (default), 'orange', 'green', 'red', 'purple', 'grey' and 'black'

\usepackage{lipsum} % Used for inserting dummy 'Lorem ipsum' text into the template
\usepackage{multicol}

\usepackage[scale=0.90]{geometry} % Reduce document margins
\setlength{\hintscolumnwidth}{3cm} % Uncomment to change the width of the dates column
%\setlength{\makecvtitlenamewidth}{10cm}
% For the 'classic' style, uncomment to adjust the width of the space allocated to your name


%----------------------------------------------------------------------------------------
%	NAME AND CONTACT INFORMATION SECTION
%----------------------------------------------------------------------------------------

\firstname{David} % Your first name
\familyname{Heymes} % Your last name
% All information in this block is optional, comment out any lines you don't need
\title{Dr. rer. nat.}
\address{Ehrlichweg 24 }{70565 Stuttgart}
\mobile{+491785631483}
\mobile{+447442552109}
\email{david@heymes.eu}
\photo[140pt][0.6pt]{photo/Heymes_David_1.jpg} % The first bracket is the picture height, the second is the thickness of the frame around the picture (0pt for no frame)

%----------------------------------------------------------------------------------------

\begin{document}
\thispagestyle{empty}

\makecvtitle % Print the CV title
%
\section{\textbf{Profil}}
%
Als Doktor der Physik mit mehrj\"ahriger Erfahrung in der universit\"aren Forschung suche ich nun nach neuen Herausforderungen au\ss erhalb der Grundlagenforschung. Ich arbeite mit Begeisterung an Projekten, in denen ich meine vielf\"altigen F\"ahigkeiten benutzen und weiter entwickeln kann. Meine erfolgreiche Karriere in der Wissenschaft zeigt meine Qualifikation, komplizierte Probleme zu durchdenken und mit Hilfe moderner Technologien effiziente L\"osungen zu finden.       
%
%
%----------------------------------------------------------------------------------------
%	POSITIONS
%----------------------------------------------------------------------------------------
%

\section{\textbf{Berufliche Laufbahn}}
%
\cventry{2015--2017}{Wissenschaftlicher Mitarbeiter}{}{}{University of Cambridge, Centre for precision studies, Cambridge, UK (www.precision.hep.phy.cam.ac.uk)}{
 \begin{itemize}
 \item Pr\"azisionsrechnungen f\"ur Messungen am Large Hadron Collider am CERN, basierend auf Monte-Carlo-Simulationen (C++, Python)
 \item Entwicklung neuer Algorithmen f\"ur numerische Pr\"azisionsrechnungen in der Teilchenphysik
   (Mathematica, Python, C++)
 \item Statistische Auswertung simulierter Daten
 \item Visualisierung und Kommunikation der Ergebnisse auf internationalen Konferenzen    
\end{itemize}
}
%
%
%----------------------------------------------------------------------------------------
%	EDUCATION SECTION
%----------------------------------------------------------------------------------------

\section{\textbf{Ausbildung}}
%
\cventry{2012--2015}{Doktorand Theoretische Physik}{RWTH Aachen}{Note: Magna cum laude (sehr gut)}{}{Dissertation: \textit{A general subtraction scheme for next-to-next-to-leading order computations in perturbative Quantum Chromodynamics}
\begin{itemize}
\item Entwicklung eines Algorithmus f\"ur Pr\"azisionsvorhersagen in der Teilchenphysik an Teilchenbeschleunigern       
\item Implementierung einer modernen Simulationssoftware in der Teilchenphysik  (C++)  
\item Ver\"offentlichung des Algorithmus in renommierter Fachzeitschrift (Nucl. Phys. B)
\end{itemize}
}
%
\cventry{2011--2012}{Master of Science - Physik}{RWTH Aachen}{Note: 1.1 (mit Auszeichnung)}{}{Thesis: \textit{Double-real radiation for dijet production}
\begin{itemize}
\item Monte-Carlo-Simulationen von Teilchenkollisionen
\item Analytische, computergest\"utzte Rechnungen in Feldtheorien (Mathematica)
\end{itemize}
}
%
\cventry{2010--2011}{Master Studium - Physik}{EPF Lausanne, Schweiz}{}{}{Erasmus Programm
}
%
\cventry{2007--2010}{Bachelor of Science - Physics}{RWTH Aachen}{Note: 1.4 (sehr gut)}{}{Thesis: Casimir effect in materials with negative refractive index
}
%
\cventry{1997--2006}{Abitur}{St. Wolfhelm Gymnasium, Schwalmtal}{Note: 1.7}{}{}
%
%\cventry{1993--1997}{Primary school}{GGS, Elmpt}{}{}{}

%

%
%----------------------------------------------------------------------------------------
%	COMPUTER SKILLS SECTION
%----------------------------------------------------------------------------------------
%
\section{\textbf{Programmiersprachen und EDV-Kenntnisse}}
%
\cvitem{C++}{Entwicklung moderner, modularer Simulationssoftware in der Teilchenphysik, Monte-Carlo-Simulationen}
%
\cvitem{Python, SciPy}{Statistische Analyse (simulierter) Daten, Visualisierung von Daten, Verarbeitung gro\ss er Datenmengen}
%
\cvitem{Mathematica}{Analytische Rechnungen in Quantenfeldtheorien, Visualisierung von Daten}
%
\cvitem{High Performance Computing}{Simulationsrechnungen auf dem Computercluster der Universit\"at Cambridge}
\cvitem{Sonstige}{Java, Fortran, MATLAB, Linux, shell-scripting, Latex, Windows, Office}

%
%
%----------------------------------------------------------------------------------------
%	PRAKTIKA
%----------------------------------------------------------------------------------------
%
\section{\textbf{Weitere T\"atigkeiten}}
%
%
\cventry{2016}{Mitglied im internationalen Planungsteam}{Konferenz: QCD@LHC 2016, Z\"urich, Schweiz}{}{}{
\begin{itemize}
\item Organisation des Fachbereiches  "Heavy Quarks'' im Rahmen der Konferenz QCD@LHC 2016
\end{itemize}
}
%
%
\cventry{2011--2015}{Lehre}{Institut f\"ur theoretische Teilchenphysik, RWTH Aachen}{}{}{
\begin{itemize}
\item Begleitender \"Ubungsgruppenleiter f\"ur Bachelor- und Mastervorlesungen (u.a. Klassische Mechanik, Klassische Elektrodynamik, Statistische Physik, Relativistische Quantenmechanik)
\end{itemize}
}
%
%
\cventry{2009--2010}{Studentische Hilfskraft}{Institut f\"ur Technologie optischer Systeme RWTH Aachen}{}{}{
\begin{itemize}
\item Vermessung von Freiformoptiken
\item Analyse und Visualisierung der gemessenen Daten (python, MATLAB)
\end{itemize}
}

%

%
%----------------------------------------------------------------------------------------
%	Publikationen
%----------------------------------------------------------------------------------------
%
\section{\textbf{Publikationen} (peer-reviewed)}
%
5 in internationalen Fachzeitschriften ver\"offentlichte, wissenschaftliche Aufs\"atze
%
\newline
\newline
Ausgew\"ahlte Publikationen (Stand: Mai 2017):
\newline
\newline
%
\cvitem{}{M.~Czakon, P.~Fiedler, \textbf{D.~Heymes} and A.~Mitov,
  ``NNLO QCD predictions for fully-differential top-quark pair production at the Tevatron,''
  JHEP 1605 (2016) 034, Zitate: 40}
%

%
\cvitem{}{M.~Czakon, \textbf{D.~Heymes} and A.~Mitov,
  ``High-precision differential predictions for top-quark pairs at the LHC,''
  Phys.\ Rev.\ Lett.\  116 (2016) no.8,  082003, Zitate: 84}
%
%
\cvitem{}{M.~Czakon and \textbf{D.~Heymes},
  ``Four-dimensional formulation of the sector-improved residue subtraction scheme,''
  Nucl.\ Phys.\ B  890 (2014) 152, Zitate: 37}
%
%
%----------------------------------------------------------------------------------------
%	Vortr\"age
%----------------------------------------------------------------------------------------
%
\section{\textbf{Fachvortr\"age}}
%
11 eingeladene Fachvortr\"age auf internationalen Konferenzen, 3 Fachseminare an physikalischen Instituten, 1 Vortrag f\"ur Gymnasiasten der 10. und 11. Jahrgangsstufe  \"uber Teilchenphysik
%
\newline
\newline
Ausgew\"ahlte Vortr\"age und Seminare:
\newline
\newline
%
\cvitem{Oktober 2016}{"High $p_T$ top-pair predicitons including QCD and EW corrections'', Precision theory for precise measurements 2016, Quy-Nhon, Vietnam}
%
\cvitem{August 2015}{"Latest developments in differential distributions at NNLO'', Large Hadron Collider Physics (LHCP) 2015, St. Petersburg, Russland}
%
\cvitem{Juni 2014}{"General formulation of the sector-improved residue subtraction'', Loopfest 2014, City Tech New York City, USA}
%
%
\cvitem{November 2016}{Seminar der Teilchenphysik, Deutsches Elektronen-Synchrotron (DESY), Hamburg, Deutschland}
%
\cvitem{Februar 2016}{Seminar der Teilchenphysik, Max-Planck-Institut f\"ur Physik, M\"unchen, Deutschland}
%
%
%----------------------------------------------------------------------------------------
%	SPRACHEN
%----------------------------------------------------------------------------------------
%
\section{\textbf{Sprachen}}
\cvlanguage{Deutsch}{Muttersprache}{}
\cvlanguage{Englisch}{verhandlungssicher}{C1}
\cvlanguage{Franz\"osisch}{konversationssicher}{B1}

%
%
%----------------------------------------------------------------------------------------
%	INTERESSEN
%----------------------------------------------------------------------------------------
%
\section{\textbf{Interessen}}
%
\cvitem{}{Mountainbike, (Freiwasser-) Schwimmen, Fahrradreisen, Literatur}
%
\end{document}


%
%
%----------------------------------------------------------------------------------------
%	Teaching Experiences
%----------------------------------------------------------------------------------------
%
%\section{\textbf{Teaching Experiences}}
%
%\cvitem{WS 2014--2015}{Exercise class for Bachelor level lecture, \textit{Theoretical Physics II: Classical electrodynamics}}
%
%\cvitem{WS 2013--2014}{Exercise class for Bachelor level lecture, \textit{Theoretical Physics IV: Statistical Physics}}
%
%\cvitem{SS 2013}{Exercise class for Bachelor level lecture, \textit{Relativistic Quantum Mechanics}}
%
%\cvitem{SS 2012}{Exercise class for Bachelor level lecture, \textit{Theoretical Physics I: Classical mechanics}}
%
%\cvitem{SS 2011}{Exercise class for Bachelor level lecture, \textit{Mathematics for Enguneers: Analysis I}}
%
%----------------------------------------------------------------------------------------
%	AWARDS SECTION
%----------------------------------------------------------------------------------------
%
%\section{\textbf{Auszeichnungen}}
%
%\cvitem{June 2013}{"Springorium-Denkm\"unze", Anerkennung f\"ur die mit Ausyeichnung bestandene Masterpr\"ufung}
%
%----------------------------------------------------------------------------------------
%	Internships
%----------------------------------------------------------------------------------------
%
%\subsection{Internships}
%
%\cvitem{2011}{Internship Forschungszentrum Juelich}
%
%\cvitem{2009-2010}{Student Research Assistant at the Fraunhofer Institute of Technology of Optical Systems, RWTH Aachen}
%
%----------------------------------------------------------------------------------------
%	LANGUAGES SECTION
%----------------------------------------------------------------------------------------
